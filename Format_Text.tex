
\renewcommand{\baselinestretch}{1.25}\normalsize

\renewcommand{\chaptermark}[1]{\markboth{\MakeUppercase{\thechapter\; #1}}{}}
\renewcommand{\sectionmark}[1]{\markright{\MakeUppercase{\thesection \; #1}}}

\fancyhead{}
\fancyfoot{}

% Abhaengig von Drucklayout
\ifthenelse{\equal{\Drucklayout}{Zweiseitig}}{
	\fancyhead[RO]{\scriptsize \rightmark}
	\fancyhead[LE]{\scriptsize \leftmark}
	\fancyfoot[LE,RO]{\thepage}
	
	% Kopf- und Fu�zeilen der Kapitelseiten
	\fancypagestyle{plain}{
   		\fancyhf{}
   		\renewcommand{\headrulewidth}{0pt}
   		\renewcommand{\footrulewidth}{0.75pt}
   		\fancyhead{}
   		\fancyfoot[LE,RO]{\thepage}
	}
}{
	\fancyhead[R]{\scriptsize \leftmark}
	\fancyfoot[R]{\thepage}
	
	% Kopf- und Fu�zeilen der Kapitelseiten
	\fancypagestyle{plain}{
   		\fancyhf{}
   		\renewcommand{\headrulewidth}{0pt}
   		\renewcommand{\footrulewidth}{0.75pt}
   		\fancyhead{}
   		\fancyfoot[R]{\thepage}
	}
}


\renewcommand{\headrulewidth}{.75pt}
\renewcommand{\footrulewidth}{.75pt}


% Bezeichnungen und Nummerierungsstile
\renewcommand{\figurename}{Abbildung}
%\renewcommand{\chaptername}{Abschnitt}

% Referenzieren von Subcaptions
\captionsetup[subfigure]{labelformat=simple}
\renewcommand\thesubfigure{(\alph{subfigure})}

% Nummerierung der Fussnoten
\counterwithout{footnote}{chapter}
\counterwithout{figure}{chapter}
\counterwithout{table}{chapter}